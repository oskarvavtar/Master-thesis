\PassOptionsToPackage{prologue,dvipsnames}{xcolor}
\documentclass{beamer}
\usepackage[utf8]{inputenc}
\usepackage[english]{babel}
\usepackage{lmodern}

\usepackage{amsthm}
\usepackage{amsmath, amssymb, amsfonts}
\usepackage{mathrsfs}
\usepackage{graphicx}
\graphicspath{ {./Slike/} }
\usepackage{bbm}
\usepackage[dvipsnames]{xcolor}
\usepackage{tikz}
\usepackage{caption}
\usepackage{subcaption}

\newcommand{\A}{\mathcal{A}}
\renewcommand{\AA}{\mathsf{A}}
\newcommand{\B}{\mathcal{B}}
\newcommand{\AB}{\mathsf{B}}
\newcommand{\BB}{\mathscr{B}}
\newcommand{\BBB}{\mathbb{B}}
\renewcommand{\c}{\mathsf{c}}
\newcommand{\E}{\mathsf{E}}
\newcommand{\F}{\mathcal{F}}
\newcommand{\G}{\mathcal{G}}
\renewcommand{\H}{\mathcal{H}}
\newcommand{\M}{\mathcal{M}}
\newcommand{\MM}{\overline{\mathfrak{M}}}
\newcommand{\N}{\mathbb{N}}
\renewcommand{\r}{\mathrm{r}}
\newcommand{\Z}{\mathbb{Z}}
\newcommand{\ZZ}{\mathcal{Z}}

\newcommand{\set}[1]{\left\{#1\right\}}
\newcommand{\oklepaj}[1]{\left(#1\right)}
\newcommand{\oglati}[1]{\left[#1\right]}
\newcommand{\ra}{\rightarrow}
\newcommand{\pika}{\boldsymbol{\cdot}}
\newcommand{\1}{\mathbbm{1}}
\renewcommand{\sp}[1]{\langle #1\rangle}
\newcommand{\ind}{\perp\!\!\!\!\perp}
\renewcommand{\c}{\mathsf{c}}
\newcommand{\5}{\vspace{0.5cm}}
\newcommand{\3}{\vspace{0.3cm}}

\usecolortheme{rose}
\useinnertheme[shadows]{rounded}
\useoutertheme{infolines}
\beamertemplatenavigationsymbolsempty

\usepackage{palatino}
\usefonttheme{serif}

\theoremstyle{definition}
\newtheorem*{form}{Formalities}
\newtheorem{ex}{Example}
\newtheorem{thm}{Theorem}[section]
\newtheorem{prop}[thm]{Proposition}
\newtheorem*{sol}{Solution}
\newtheorem*{dis}{Disclaimer}
\newtheorem{df}[thm]{Definition}
\newtheorem{rem}[thm]{Remark}
\newtheorem{fct}[thm]{Fact}
\newtheorem{lem}[thm]{Lemma}
\newtheorem*{sett}{Setting}
\newtheorem{cor}{Corollary}
\newtheorem*{que}{Question}
\newtheorem*{conj}{Conjecture}
\newtheorem*{rec}{Recipe}
\newtheorem*{goal}{Goal}
\newtheorem*{approach}{Approach}

\begin{document}

% #################################################################################################

\title[Scientific Talk]{Absence of Phase Transitions and Preservation of Gibbs Property Under Renormalization \\\vspace{0.4cm} \small{Scientific Talk}}
\author[Oskar Vavtar]{Oskar Vavtar \\\vspace{0.2cm} \small{Supervisor: Dr.~Evgeny A.~Verbitskiy}}
\date{June 21, 2024}
\institute[LU]{Leiden University, \\ Mathematical Institute}

\begin{frame}
	\titlepage
\end{frame}

% #################################################################################################
% #################################################################################################
% #################################################################################################

\section{Introduction}

% #################################################################################################

\begin{frame}
\frametitle{Preliminaries}
Consider finite alphabet $\A$ and write $\Omega=\A^{\Z^d}$.
\begin{df}[Interactions and Hamiltonians]
\begin{itemize}
	\item[(1)] \textbf{Interaction} is a collection of maps $\Phi=(\Phi_\Lambda)_{\Lambda\Subset\Z^d}$, where
	$$\Phi_\Lambda(\omega) ~=~ \Phi_\Lambda(\omega(x):x\in\Lambda), \quad \omega\in\Omega.$$
	We say $\Phi$ is \textbf{uniformly absolutely convergent} (UAC), $\Phi\in\BB^1(\Omega)$, if
	$$\sup_{x\in\Z^d}\sum_{\Lambda\ni x}\|\Phi_\Lambda\|_\infty ~<~ \infty.$$
	\item[(2)] For $\Phi\in\BB^1(\Omega)$, we consider \textbf{Hamiltonians} $\H=(\H_{\Lambda})_{\Lambda\Subset\Z^d}$,
	$$\H_\Lambda(\omega) ~=~ \sum_{\Delta\cap\Lambda\neq\emptyset}\Phi_\Delta(\omega), \quad \omega\in\Omega.$$
\end{itemize}
\end{df}
\end{frame}

\begin{frame}
\frametitle{Preliminaries}
\begin{df}[Specification and Gibbs measure]
\begin{itemize}
	\item[(1)] For $\Phi\in\BB^1(\Omega)$, \textbf{specification} $\gamma=(\gamma_\Lambda)_{\Lambda\Subset\Z^d}$, is given so that $\gamma_\Lambda(\pika|\pika):\F_\Lambda\times\Omega_{\Lambda^\c}\ra(0,1)$,
	$$\gamma_\Lambda(\omega_\Lambda|\xi_{\Lambda^\c}) ~=~ \frac{1}{\ZZ_\Lambda^\xi}\exp(-\H_\Lambda(\omega_\Lambda\xi_{\Lambda^\c})),$$
	where $\ZZ_\Lambda^\xi$ is the normalization constant.
	\item[(2)] $\mu\in\M_1(\Omega)$ is a \textbf{Gibbs measure} on $\Omega$ consistent with $\Phi$ ($\mu\in\G_\Omega(\Phi)$) if for each $\Lambda\Subset\Z^d$,
	$$\mu(\omega_\Lambda|\omega_{\Lambda^\c}) ~=~ \gamma_\Lambda^\Phi(\omega_\Lambda|\omega_{\Lambda^\c}) \quad \text{for}~\mu\text{-a.a.}~\omega\in\Omega.$$
\end{itemize}
\end{df}
\end{frame}

\begin{frame}
\frametitle{Preliminaries}
\begin{rem}
\begin{itemize}
	\item[(i)] For $\Lambda\Subset\Z^d$ and $\xi_{\Lambda^\c}$ fixed, $\gamma_{\Lambda}(\pika|\xi_{\Lambda^\c})$ is a probability measure on $\Omega_\Lambda=\A^\Lambda$. This allows for construction of Gibbs measures via weak limits.
	\item[(ii)] While $\G_{\Omega}(\Phi)\neq \emptyset$, we don't necessarily have that $|\G_{\Omega}(\Phi)|=1$. How and when this happens is an important subject in statistical mechanics.
\end{itemize}
\end{rem}
\end{frame}

% #################################################################################################

\begin{frame}
\frametitle{Characterization of Gibbsianity}
\begin{prop}
Let $\mu\in\M_1(\Omega)$. The following are equivalent:
	\begin{itemize}
		\item[(i)] $\mu$ is Gibbs
		\item[(ii)] $\mu$ has the following properties:
		\begin{itemize}
			\item[(a)] \textbf{uniform non-nullness}: $\forall\Lambda\Subset\Z^d$ $\exists \alpha_\Lambda,\beta_\Lambda\in(0,1)$ s.t.
			$$\alpha_\Lambda ~\leq~ \mu_\Lambda(\omega_\Lambda|\xi_{\Lambda^\c}) ~\leq~ \beta_\Lambda, \quad \forall \omega,\xi\in\Omega$$
			\item[(b)] \textbf{quasilocality}: writing $\BBB_n=[-n,n]^d\cap\Z^d$, $\forall \Lambda\Subset\Z^d$,
			$$\sup_{\omega}\sup_{\xi,\zeta}|\mu(\omega_\Lambda|\omega_{\BBB_n\setminus\Lambda}\xi_{\BBB_n^\c\setminus\Lambda})-\mu(\omega_\Lambda|\omega_{\BBB_n\setminus\Lambda}\zeta_{\BBB_n^\c\setminus\Lambda})| ~\ra~ 0.$$ 
		\end{itemize}
	\end{itemize}
\end{prop}
\end{frame}

% #################################################################################################

\begin{frame}
\frametitle{Renormalization group}
\texttt{To be done later} \\\vspace{1cm}
\textit{Dilemma: talk about renormalization or just fuzzy Gibbs measures?}
\end{frame}

% #################################################################################################
% #################################################################################################
% #################################################################################################

\section{Fuzzy Gibbs framework}

% #################################################################################################

\begin{frame}
\frametitle{Fuzzy Gibbs measures}
Consider $\Omega=\A^{\Z^d}$, with $\A$ finite, as before. \\\vspace{0.5cm}
Let $\B$ be another alphabet, with $|\B|\leq|\A|$. Write $\Sigma=\B^{\Z^d}$. \\\vspace{0.5cm}
We consider a surjection $\pi:\A\ra\B$, which we call a \textbf{fuzzy map}.\footnote{$\pi$ induces a surjection $\Omega\ra\Sigma$ which we denote by the same letter}
\begin{df}
A fuzzy Gibbs measure $\nu$ on $\Sigma$ is defined as
$$\nu ~=~ \mu\circ\pi^{-1},$$
where $\mu$ is some Gibbs measure on $\Omega$.
\end{df}\vspace{0.2cm}
\textcolor{red}{\textbf{Question:}} when is $\nu$ Gibbsian?
\end{frame}

% #################################################################################################

\begin{frame}
\frametitle{Hidden phase transitions}
We can partition $\Omega$ w.r.t.~$\pi$ as follows:\\\vspace{0.3cm}
pick $\sigma\in\Sigma$ and define
$$\Omega_\sigma ~=~ \pi^{-1}(\sigma);$$
we call sets $\set{\Omega_\sigma:\sigma\in\Sigma}$ \textbf{fibres}.
\begin{df}[Hidden phase transition]
We say that a \textbf{hidden phase transition} occurs on $\Omega_\sigma$ if
$$|\G_{\Omega_\sigma}(\Phi)| ~>~ 1.$$
If $|\G_{\Omega_\sigma}(\Phi)|=1$ for all $\sigma\in\Sigma$, we talk about \textbf{absence of hidden phase transitions}.

\end{df}
\end{frame}

\begin{frame}
\frametitle{Hidden phase transitions}
\begin{prop}[Sufficient condition]
In the absence of hidden phase transitions, $\nu=\mu\circ\pi^{-1}$ is Gibbsian.
\end{prop}\vspace{0.3cm}
The following conjecture (stated informally here) was established by van Enter, Fern\'andez and Sokal:
\begin{conj}[van Enter-Fern\'andez-Sokal hypothesis, \cite{EFS},\cite{Ber}]
The fuzzy Gibbs measure is not Gibbsian \textit{if and only if}
\begin{itemize}
	\item[(i)] $\exists\sigma\in\Sigma:|\G_{\Omega_\sigma}(\Phi)|>1$, i.e., a hidden phase transition occurs, and
	\item[(ii)] one can pick different phases of $\G_{\Omega_\sigma}(\Phi)$ by varying boundary conditions.
\end{itemize}
\end{conj}
\end{frame}

% #################################################################################################

\begin{frame}
\frametitle{Construction of conditional measures}
\textcolor{teal}{\textbf{Goal:}} construct distribution of $\omega$, conditional on $\pi(\omega)=\sigma$
\begin{df}
Given $B\subseteq\Sigma$ measurable with $\nu(B)>0$, define
$$\mu^B ~=~ \mu(\pika|\pi^{-1}(B)).$$
\end{df}
One can consider a net of conditional measures $\mu^B$ (on $\Omega$), indexed with pairs $(V,B)$, where $V$ is an open neighbourhood of $\sigma$ and $B\subseteq V:\nu(B)>0$. \\\vspace{0.5cm}
Write $\MM_\sigma$ for accumulation points of the above net, as open neighbourhoods ($V$) ``approach'' $\sigma$.
\end{frame}

\begin{frame}
\frametitle{Tjur points}
\begin{df}
If $|\MM_\sigma|=1$ for a given $\sigma\in\Sigma$, denote by $\mu^\sigma$ the only member of $\MM_\sigma$, the limit of the corresponding net. In this case, we say that $\sigma$ is a \textbf{Tjur point}.
\end{df}\vspace{0.3cm}
One can restate the previously presented conjecture as follows:
\begin{conj}[van Enter-Fern\'andez-Sokal hypothesis, \cite{Ber}]
The fuzzy Gibbs measure is Gibbsian \textit{if and only if} $|\MM_\sigma|=1$ for all $\sigma\in\Sigma$, i.e., all points are Tjur.
\end{conj}\vspace{0.3cm}
\begin{prop}[Berghout, Verbitskiy, \cite{Ber}]
Direction $(\Leftarrow)$ holds.
\end{prop}
\end{frame}

\begin{frame}
\frametitle{Tjur points: sufficient condition revisited}
\begin{prop}
$\MM_\sigma\neq 0$, each member is a probability measure supported on $\Omega_\sigma$. If $\mu\in\G_\Omega(\Phi)$, then 
$$\MM_\sigma ~\subseteq~ \G_{\Omega_\sigma}(\Phi).$$
\end{prop}\vspace{0.2cm}
\begin{cor}
Absence of phase transitions implies Gibbsianity of $\nu=\mu\circ\pi^{-1}$.
\end{cor}\vspace*{0.2cm}
\begin{rem}
By demonstrating the absence of phase transitions, we not only obtain Gibbsianity of the fuzzy Gibbs measure, but also verify that the example doesn't contradict the unproven direction of the van Enter-Fern\'andez-Sokal hypothesis.
\end{rem}
\end{frame}

% #################################################################################################
% #################################################################################################
% #################################################################################################

\section{Fuzzy Potts model}

% #################################################################################################

\begin{frame}
\frametitle{Classical Potts model}
Write $\E^d$ for the (nearest-neighbour) edge set of $\Z^d$ and
$$\E_\Lambda=\set{\sp{x,y}\in\E^d:x,y\in\Lambda}, \quad \partial\E_\Lambda=\set{\sp{x,y}\in\E^d:x\in\Lambda,y\notin\Lambda}.$$
\begin{df}[Interaction of Potts model]
The interaction of $q$-state Potts model $\Phi_{\beta,q}$ is given by
$$\Phi_{\Lambda;\beta,q}(\omega) ~=~ \begin{cases}
2\1_{\set{\omega(x)\neq\omega(y)}}-1, ~& \Lambda=\set{x,y}:x\sim y, \\
0, ~&\text{otherwise}.
\end{cases}$$
\end{df}
Hamiltonians are thus given by
$$\H_{\Lambda;\beta,q}(\omega) ~=~ \sum_{\sp{x,y}\in\E_\Lambda\cup\partial\E_\Lambda}(2\1_{\set{\omega(x)\neq\omega(y)}}-1).$$
\end{frame}

\begin{frame}
\frametitle{Classical Potts model: phase transition}
Write $\Omega=\set{1,\ldots,q}^{\Z^d}$.\vspace{0.3cm}
\begin{thm}
For each $q\geq 2$ and $d\geq 2$, there exists $\beta_c(d,q)\in(0,\infty)$, such that 
\begin{itemize}
	\item[(i)] for $\beta<\beta_c(d,q)$, $|\G_{\Omega}(\Phi_{\beta,q})|=1$,
	\item[(ii)] for $\beta>\beta_c(d,q)$, $\G_{\Omega}(\Phi_{\beta,q})$ contains $q$ distinct mutually singular measures .
\end{itemize}
\end{thm}\vspace{0.3cm}
Mutually singular measures in (ii) are precisely measures $\mu_{\beta,q}^{\Z^d,1},\ldots,\mu_{\beta,q}^{\Z^d,q}$, corresponding to constant boundary conditions $\mathsf{1},\ldots,\mathsf{q}$.
\end{frame}

% #################################################################################################

\begin{frame}
\frametitle{Fuzzy Potts model}
Let $1<s<q$ and $\r=(r_1,\ldots,r_s)$, such that $r_1+\ldots+r_s=q$.
\begin{df}
Fuzzy Potts map $\pi_\r:\set{1,\ldots,q}\ra\set{1,\ldots,s}$ is given by
$$\pi_\r(a) ~=~ \begin{cases}
1: ~&1\leq a\leq r_1, \\
2: ~&r_1+1<a\leq r_1+r_2, \\
\ldots \\
n: ~& r_1+\ldots+r_{n-1}<a\leq r_1+\ldots r_n, \\
\ldots \\
s: ~& r_1+\ldots+r_{s-1}<a\leq q.
\end{cases}$$
Fuzzy Gibbs measure corresponding to $\mu_{\beta,q}^{\Z^d,\xi}$ is given by
$$\nu_{\beta,q}^{\Z^d,\xi} ~=~ \mu_{\beta,q}^{\Z^d,\xi}\circ\pi_{\r}^{-1}.$$
\end{df}
\end{frame}

\begin{frame}
Write $r^*=\min(\set{r_1,\ldots,r_s}\cap\N_{\geq 2})$
\frametitle{Fuzzy Potts model: Gibbsianity}
\begin{thm}[H\"aggstr\"om, \cite{Hag}]
Let $d\geq 2$, $q\geq 3$ and $\xi\in\set{\emptyset,\mathsf{1},\ldots,\mathsf{q}}$; consider fuzzy Potts measure $\mu_{\beta,q}^{\Z^d,\xi}$.
\begin{itemize}
	\item[(i)] For each $\beta<\beta_c(d,r^*)$, $\nu_{\beta,q}^{\Z^d,\xi}$ is a Gibbs measure.
	\item[(ii)] For each $\beta>\frac{1}{2}\log\frac{1+(r^*-1)p_c(d)}{1-p_c(d)}$, $\nu_{\beta,q}^{\Z^d,\xi}$ is \textit{not} a Gibbs measure.\footnote{$p_c(d)$ = critical probability for Bernoulli percolation on $\Z^d$}
\end{itemize}
\end{thm}\vspace{0.3cm}
\textcolor{teal}{\textbf{Goal:}} provide an alternative proof of (i), using absence of hidden phase transitions.
\end{frame}

% #################################################################################################

\begin{frame}
\frametitle{Idea of alternative proof}
\textcolor{teal}{\textbf{Want to show:}} for each $\sigma\in\set{1,\ldots,s}^{\Z^d}$, $|\G_{\Omega_\sigma}(\Phi_{\beta,q})|=1$. \\\vspace{0.5cm}
\textcolor{orange}{\textbf{Notice:}} 
$$\Omega_\sigma ~=~ \prod_{x\in\Z^d}\pi^{-1}(\sigma(x));$$
write, for $j=1,\ldots,s$, 
$$\AA_j ~=~ \pi^{-1}(j) ~=~ \set{r_1+\ldots+r_{j-1}+1,\ldots,r_1+\ldots+r_j}$$
and
$$U_j ~=~ \set{x\in\Z^d:\sigma(x)=j}.$$
Then,
$$\Omega_\sigma ~=~ \prod_{x\in\Z^d}\begin{cases}
\AA_1, ~&x\in U_1, \\
\ldots \\
\AA_s, ~&x\in U_s
\end{cases} ~=:~ \bigotimes_{j=1}^s \AA_j^{U_j}.$$
\end{frame}

\begin{frame}
\frametitle{Idea of alternative proof}
It is enough to show that:\vspace{0.3cm}
\begin{itemize}
	\item[(i)] If $\beta$ is such that 
	$$|\G_{\AA_j^{\Z^d}}(\Phi_{\beta,|\AA_j|})| ~=~ 1, \quad \forall j=1,\ldots,s,$$
	then
	$$|\G_{\AA_j^{U_j}}(\Phi_{\beta,|\AA_j|})| ~=~ 1, \quad \forall j=1,\ldots,s.$$
	\item[(ii)] If 
	$$|\G_{\AA_j^{U_j}}(\Phi_{\beta,|\AA_j|})| ~=~ 1, \quad \forall j=1,\ldots,s,$$
	then
	$$|\G_{\otimes_j\AA_j^{U_j}}(\Phi_{\beta,q})| ~=~ 1.$$
\end{itemize}
\end{frame}

\begin{frame}
\frametitle{Idea of alternative proof}
\textcolor{orange}{\textbf{Clear:}} enough to show above for $s=2$, induction takes care of the rest. Thus sufficient to prove:
\begin{prop}[Part I]
Let $U\subset\Z^d$ and $q\in\N_{\geq 2}$. For $\beta<\beta_c(d,q)$,
$$|\G_{\set{1,\ldots,q}^U}(\Phi_{\beta,q})| ~=~ 1.$$
\end{prop}
\begin{prop}[Part II]
Let $\Z^d=U\sqcup V$ and $\AA\cap\AB=\emptyset$. If $\beta$ is such that
$$|\G_{\AA^U}(\Phi_{\beta,|\AA|})| ~=~ |\G_{\AB^V}(\Phi_{\beta,|\AB|})| ~=~ 1,$$
then
$$|\G_{\AA^U\otimes\AB^V}(\Phi_{\beta,|\AA|+|\AB|})| ~=~ 1.$$
\end{prop}
\end{frame}

% #################################################################################################
% #################################################################################################
% #################################################################################################

\section{Spin-flip dynamics}

% #################################################################################################

\begin{frame}
\frametitle{Spin-flip dynamics: general model}
\textcolor{teal}{\textbf{Idea:}} Pick initial configuration $\omega_0\in\set{-1,+1}^{\Z^d}$ according to some Gibbs measure and randomly flip spins as time runs. \\\vspace{0.5cm}
\textcolor{orange}{\textbf{Question:}} Having obtained $(\omega_t)_{t\geq 0}$, when is $\mathrm{Law}(\omega_t)$ Gibbsian?
\end{frame}

\begin{frame}
\frametitle{Spin-flip dynamics: general model}
Let $\Omega_0=\set{-1,+1}^{\Z^d}$.\\\vspace{0.5cm}
Pick $\mu\in\G_{\Omega_0}$ and draw $\omega_0\sim\mu$.
\end{frame}

% #################################################################################################
% #################################################################################################
% #################################################################################################

\section*{~}

\begin{frame}
	\begin{thebibliography}{++}

\bibitem[Ber20]{Ber} S.~Berghout. \textit{Gibbs Processes and Applications}. Ph.D.~thesis. Leiden University, 2020.

\bibitem[vEFS93]{EFS} A.C.D.~van Enter, R.~Fern\'andez, A.D.~Sokal. \textit{Regularity properties and pathologies of position-space renormalization-group transformations: Scope and limitations of Gibbsian theory}. J.~Statist.~Phys.~72 (1993), no.~5-6, 879-1167.

\bibitem[H\"ag03]{Hag} O.~H\"aggstr\"om. \textit{Is the fuzzy Potts model Gibbsian?} Ann.~I.~H.~Poincar\'e 39 (2003), no.~5, 891-917.

	\end{thebibliography}
\end{frame}

% #################################################################################################

\end{document}
