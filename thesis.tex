 \documentclass[12pt]{article}
\usepackage[utf8]{inputenc}
\usepackage[english]{babel}

\usepackage{amsthm}
\usepackage{amsmath, amssymb, amsfonts}
\usepackage{mathrsfs}
\usepackage{titling}
\usepackage{hyperref}
\usepackage{url}
\usepackage{bbm}
\usepackage{xcolor}
\usepackage{graphicx}
\graphicspath{ {./Slike/} }
\usepackage{subcaption}

\usepackage{geometry}
\geometry{
 a4paper,
 %total={170mm,257mm},
 left=30mm,
 right=30mm,
 top=20mm,
 bottom=20mm
}

\renewcommand{\H}{\mathcal{H}}
\newcommand{\Loc}{\mathcal{L}}
\renewcommand{\L}{\mathbb{L}}
\newcommand{\N}{\mathbb{N}}
\renewcommand{\P}{\mathbb{P}}
\newcommand{\R}{\mathbb{R}}
\renewcommand{\r}{\mathrm{r}}
\newcommand{\Z}{\mathbb{Z}}
\newcommand{\ZZ}{\mathcal{Z}}

\newcommand{\set}[1]{\left\{#1\right\}}
\newcommand{\oklepaj}[1]{\left(#1\right)}
\newcommand{\oglati}[1]{\left[#1\right]}
\newcommand{\ra}{\rightarrow}
\newcommand{\pika}{\boldsymbol{\cdot}}
\newcommand{\1}{\mathbbm{1}}
\renewcommand{\sp}[1]{\langle #1\rangle}
\newcommand{\ind}{\perp\!\!\!\!\perp}
\renewcommand{\c}{\mathsf{c}}
\newcommand{\supp}{\mathrm{supp}}
\newcommand{\5}{\vspace{0.5cm}}

\theoremstyle{definition}
\newtheorem*{form}{Formalities}
\newtheorem*{ex}{Example}
\newtheorem{thm}{Theorem}[section]
\newtheorem{prop}[thm]{Proposition}
\newtheorem*{sol}{Solution}
\newtheorem*{dis}{Disclaimer}
\newtheorem{df}[thm]{Definition}
\newtheorem*{rem}{Remark}
\newtheorem{lem}[thm]{Lemma}
\newtheorem{cor}[thm]{Corollary}

\setlength{\droptitle}{-2cm}
\title{\textsc{Random Polymers Near a Homogeneous Interface}\\\vspace{0.3cm}\small{Statistical Mechanics}\vspace{-0.7cm}}
\author{Oskar Vavtar}
\date{\today}

\begin{document}
\begin{center}
\Huge{\textcolor{teal}{\texttt{TITLE}}} \\
\vspace{2cm}
\Huge{\textcolor{teal}{\texttt{TITLE}}} \\
\vspace{2cm}
Leiden University \\
\vspace{2cm}
{\textsc{Absence of Phase Transitions and Preservation of Gibbs Property Under Renormalization}} \\
\vspace{2cm}
Oskar Vavtar \\
\vspace{2cm}
\Huge{\textcolor{teal}{\texttt{TITLE}}} \\
\vspace{2cm}
\Huge{\textcolor{teal}{\texttt{TITLE}}} \\
\vspace{2cm}
\Huge{\textcolor{teal}{\texttt{TITLE}}}
\end{center}
\pagebreak
\tableofcontents
\pagebreak

% #################################################################################################
% #################################################################################################
% #################################################################################################

\section{Introduction (probably better name)}

\textcolor{blue}{I assume this chapter will include a brief introduction to Gibbs measures/thermodynamical formalism (possibly including definitions of Ising and Potts model), as well as the theory of fuzzy Gibbs measures, results from Berghout's thesis.} 

\pagebreak

% #################################################################################################
% #################################################################################################
% #################################################################################################

\section{(Non-)Gibbsianity of fuzzy Potts model}

This chapter aims to introduce the fuzzy Potts model and provide an alternative, independent proof of H\"aggstr\"om's theorem \textcolor{red}{\texttt{[reference]}} about its (non-)Gibbsianity, using the results due to Berghout and Verbitskiy \textcolor{red}{\texttt{[reference]}}. Moreover, it introduces the notion of random cluster representations, a powerful tool in the theory of Potts model, which is used in the proof.

% #################################################################################################

\subsection{Fuzzy Potts model}

In this introductory section of the chapter, we define the fuzzy Potts model and state the celebrated result about its (non-)Gibbsianity, due to H\"aggstr\"om \textcolor{red}{\texttt{[reference]}}. Moreover, we explain the strategy and structure of our alternative proof of (part of) the said result. \\

Consider the Potts model with spin space $\set{1,\ldots,q}$, $q\geq 3$ integer, on lattice $\L$, say $\L=\Z^d$, which defines a model on $\Omega=\set{1,\ldots,q}^\L$. The \textit{fuzzy Potts model} is defined by considering some integer $1<s<q$, so that the spin space is $\set{1,\ldots,s}$ and the whole model defined on $\Sigma=\set{1,\ldots,s}^\L$. Moreover, we consider a vector $\r=(r_1,\ldots,r_s)\in\N^s$, such that $r_1+\ldots+r_s=q$ and define a fuzzy transformation $\pi_\r:\set{1,\ldots,q}\ra\set{1,\ldots,s}$ by putting 
$$\pi_\r(a) ~:=~ \begin{cases}
1: ~&1\leq a\leq r_1,\\
2: ~&r_1+1\leq a\leq r_1+r_2 \\
\cdots \\
n: ~&\sum_{i=1}^{n-1} r_i + 1\leq a\leq \sum_{i=1}^n r_i,\\
\cdots \\
s: ~&\sum_{i=1}^{s-1}r_i + 1\leq a\leq q,
\end{cases}$$
i.e., $\pi_a=n$ iff $a\in(\sum_{i=1}^{n-1}r_i,\sum_{i=1}^n r_i]\cap \N$, $n\in\set{1,\ldots,s}$. In other words, the entire fuzzy map $\pi=\pi_\r$ is encoded by a single $s$-vector $\r$. \\

Fixing $q\geq 2$, $\beta\geq 0$ and writing $\mu_{q,\beta}^{\Z^d,\#}$ for the Gibbs measure of the Potts model on $\set{1,\ldots,q}^{\Z^d}$ for boundary condition $\#\in\set{0,\ldots,q}$ with inverse temperature $\beta$, the fuzzy transformation $\pi_\r$ induces the fuzzy Gibbs measure 
$$\nu_{q,\beta,\r}^{\Z^d,\#} ~:=~ \mu_{q,\beta}^{\Z^d,\#}\circ\pi_\r^{-1}.$$
Of great interest is the potential Gibbsianity of such measure. \textcolor{blue}{\texttt{Something about the H\"aggstr\"om's result blahblahblah.}} Recall that for \textcolor{purple}{$q\geq 3$ and $d\geq 2$}, there exists $\beta_c(d,q)$ such that for each $\beta<\beta_c(d,q)$, $\mu_{q,\beta}^{\Z^d,0}=\ldots=\mu_{q,\beta}^{\Z^d,q}$, i.e., there is a unique Gibbs measure of the Potts model on $\set{1,\ldots,q}^{\Z^d}$ with inverse temperature $\beta$.

\begin{thm}[H\"aggstr\"om, 2003, \textcolor{red}{\texttt{[reference]}}]
Let $d\geq 2$, $q\geq 3$, $\#\in\set{0,\ldots,q}$ and $\r=(r_1,\ldots,r_s)$ with $1<s<q$, $r_1+\ldots+r_s=q$. Consider a fuzzy Gibbs measure $\nu_{q,\beta,\r}^{\Z^d,\#}=\mu_{q,\beta}^{\Z^d,\#}\circ\pi_{\r}^{-1}$.
\begin{itemize}
	\item[(i)] For each $\beta<\beta_c(d,\min_{1\leq i\leq s}r_i)$, the measure $\nu_{q,\beta,\r}^{\Z^d,\#}$ is a Gibbs measure.
	\item[(ii)] \textcolor{blue}{\texttt{The non-Gibbs part.}}
\end{itemize}
\end{thm}

% #################################################################################################
% #################################################################################################
% #################################################################################################

\end{document}